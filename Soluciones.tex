\documentclass[10pt,a4paper,openright]{book}

%% Formateo del título del documento
\title{Soluciones de los Ejercicios, MARP I}
\author{Mario Calvarro Marines}
\date{}

%% Formateo del estilo de escritura y de la pagina
\pagestyle{plain}
\setlength{\parskip}{0.35cm} %edicion de espaciado
\setlength{\parindent}{0cm} %edicion de sangría
\clubpenalty=10000 %líneas viudas NO
\widowpenalty=10000 %líneas viudas NO
\usepackage[top=2.5cm, bottom=2.5cm, left=3cm, right=3cm]{geometry} % para establecer las medidas de los margenes
\usepackage[spanish]{babel} %Para que el idioma por defecto sea español
\usepackage{ulem} % para poder subrayar entornos especiales como las secciones

%% Texto matematico y simbolos especiales
\usepackage{amsmath} %Paquetes para mates
\usepackage{amsfonts} %Paquetes para mates
\usepackage{amssymb} %Paquetes para mates
\usepackage{stmaryrd} % paquete para mates
\usepackage{latexsym} %Paquetes para mates
\usepackage{cancel} %Paquete tachar cosas
\usepackage{listings}

%% Ruta de las fotos e inclusion de las mismas
\usepackage{graphicx}
\graphicspath{{./fotos/}}

%% Inclusion de referencias cruzadas por defecto y específicas
\usepackage{hyperref}

%% Paquete para definir y utilizar colores por el documento
\usepackage[dvipsnames,usenames]{xcolor} %activar e incluir colores
    %% definicion de los colores que se van a utilizar en cada cabecera
    \definecolor{capitulos}{RGB}{60,0,0}% gama de colores de los capitulos
    \definecolor{secciones}{RGB}{95,8,5}% gama de colores de las secciones
    \definecolor{subsecciones}{RGB}{140,36,31}% gama de colores de las subsections
    \definecolor{subsubsecciones}{RGB}{188,109,79}% gama de colores de las subsubsections
    \definecolor{teoremas}{RGB}{164,56,32}% gama de colores para los teoremas
    \definecolor{demos}{RGB}{105,105,105} % gama de colores para el cuerpo de las demostraciones

%% Paquete para la edición y el formateo de capítulos, secciones...
\usepackage[explicit]{titlesec}
    %% Definición del estilo de los capítulos, secciones, etc...
    \titleformat{\chapter}[display]{\normalfont\huge\bfseries\color{capitulos}}{}{0pt}{\Huge #1}[\titlerule]
    \titleformat{\section}{\normalfont\Large\bfseries\color{secciones}}{}{0pt}{#1}
    \titleformat{\subsection}{\normalfont\large\bfseries\color{subsecciones}}{}{0pt}{\uline{#1}}
    \titleformat{\subsubsection}{\normalfont\normalsize\bfseries\color{subsubsecciones}}{}{0pt}{#1}

%% Paquete para el formateo de entornos del proyecto
\usepackage{ntheorem}[thmmarks]
    %% Definicion del aspecto de los entornos matematicos del proyecto
    \theoremstyle{break}
    \theoremheaderfont{\normalfont\bfseries\color{teoremas}}
    \theorembodyfont{\itshape}
    \theoremseparator{\vspace{0.2cm}}
    \theorempreskip{\topsep}
    \theorempostskip{\topsep}
    \theoremindent0cm
    \theoremnumbering{arabic}
    \theoremsymbol{}
    \theoremprework{\vspace{0.2cm} \hrule}
    \theorempostwork{\vspace{0.2cm}\hrule}
        \newtheorem*{defi}{Definición}

    \theoremprework{\vspace{0.25cm}}
        \newtheorem*{theo}{Teorema}

    \theoremprework{\vspace{0.25cm}}
    	\newtheorem*{coro}{Corolario}

    \theoremprework{\vspace{0.25cm}}
    	\newtheorem*{lema}{Lema}

    \theoremprework{\vspace{0.25cm}}
    	\newtheorem*{prop}{Proposición}

    \theoremheaderfont{\normalfont}
    \theorembodyfont{\normalfont\color{demos}}
    \theoremsymbol{\hfill\square}
    	\newtheorem*{demo}{\underline{Demostración}:}

    \theoremheaderfont{\normalfont}
    \theorembodyfont{\normalfont}
    	\newtheorem*{obs}{\underline{Observación}:}
    	\newtheorem*{ej}{\underline{Ejemplo}:}

%% Definicion de operadores especiales para simplificar la escritura matematica
\DeclareMathOperator{\dom}{dom}
\DeclareMathOperator{\img}{img}
\DeclareMathOperator{\rot}{rot}
\DeclareMathOperator{\divg}{div}
\newcommand{\dif}[1]{\ d#1}

%% Paquete e instrucciones para la generacion de los dibujos
\usepackage{pgfplots}
\pgfplotsset{compat=1.17}
\usepackage{tkz-fct}
\usepgfplotslibrary{fillbetween}
\usepackage{tikz,tikz-3dplot}
\tdplotsetmaincoords{80}{45}
\tdplotsetrotatedcoords{-90}{180}{-90}
\usetikzlibrary{arrows}
    %% style for surfaces
    \tikzset{surface/.style={draw=blue!70!black, fill=blue!40!white, fill opacity=.6}}

    %% macros to draw back and front of cones
    %% optional first argument is styling; others are z, radius, side offset (in degrees)
    \newcommand{\coneback}[4][]{
        %% start at the correct point on the circle, draw the arc, then draw to the origin of the diagram, then close the path
        \draw[canvas is xy plane at z=#2, #1] (45-#4:#3) arc (45-#4:225+#4:#3) -- (O) --cycle;
    }
    \newcommand{\conefront}[4][]{
        \draw[canvas is xy plane at z=#2, #1] (45-#4:#3) arc (45-#4:-135+#4:#3) -- (O) --cycle;
    }
    
    \tikzset{middlearrow/.style={decoration={markings, mark= at position 0.5 with {\arrow{#1}},},postaction={decorate}}}
    
    \usetikzlibrary{decorations.markings}
    
    \newcommand{\AxisRotator}[1][rotate=0]{
    \tikz [x=0.25cm,y=0.60cm,line width=.2ex,-stealth,#1] \draw (0,0) arc (-150:150:1 and 1);
    }
    
    \usetikzlibrary{shapes}

%% Hacer árboles
\usetikzlibrary{arrows,chains,topaths,positioning,fit,automata,shapes.misc,calc}
\usepackage{tikz-qtree}

\makeatletter

\let\old@@children\@@children
\def\@@children{\futurelet\my@next\my@@children}
\def\my@@children{%
\ifx\my@next\missing\else
\expandafter\@gobble
\fi
\expandafter\old@@children}

\makeatother

\newcommand{\missing}{ \edge[edge from parent path={(\tikzparentnode) -- ($ (\tikzparentnode) !.5! (\tikzchildnode) $)}]; \node[draw=none] {};}

\definecolor{codegreen}{rgb}{0,0.6,0}
\definecolor{codegray}{rgb}{0.5,0.5,0.5}
\definecolor{codepurple}{rgb}{0.58,0,0.82}
\definecolor{backcolour}{rgb}{0.95,0.95,0.92}

\lstdefinestyle{mystyle}{
    backgroundcolor=\color{backcolour},   
    commentstyle=\color{codegreen},
    keywordstyle=\color{magenta},
    numberstyle=\tiny\color{codegray},
    stringstyle=\color{codepurple},
    basicstyle=\ttfamily\footnotesize,
    breakatwhitespace=false,         
    breaklines=true,                 
    captionpos=b,                    
    keepspaces=true,                 
    numbers=left,                    
    numbersep=5pt,                  
    showspaces=false,                
    showstringspaces=false,
    showtabs=false,                  
    tabsize=2
}

\lstset{style=mystyle}

\begin{document}
\maketitle
\setcounter{tocdepth}{3}% para que salgan las subsubsecciones en el indice
\tableofcontents
\chapter{Hoja 1}%
\label{cha:hoja_1}

\section{Ejercicio 1. (Iker Muñoz)}%
\label{sec:ejercicio_1}
No será coste constante amortizado. Contraejemplo:
\begin{lstlisting}
fun multiapilar (p: pila, k: nat)
    mientras k > 0 hacer
        apilar (p)
        k = k - 1
    fmientras
ffun
\end{lstlisting}
Coste $O\left( k \right)$. Si consideramos la secuencia de $n$ llamadas a multiapilar tendremos:
\[
\hat{c_i} = \frac{1}{n} \sum_{i=1}^{n} c_i = \frac{1}{n} \sum_{i=1}^{n} k = k.  
\]

\section{Ejercicio 2. (Sergio García)}%
\label{sec:ejercicio_2}
Sí:
\[
\underbrace{1 \ldots \underbrace{10 \ldots 0}_{k - 1}}_{n} \rightarrow^{\text{incrementar}}  \underbrace{1 \ldots \underbrace{01 \ldots 1}_{k-1}}_{n} \rightarrow^{\text{decrementar}}  \underbrace{1 \ldots \underbrace{10 \ldots 0}_{k-1}}_{n} \rightarrow \ldots n \text{ operaciones.} 
\]
Incrementar y decrementar: $O\left( k \right) \Rightarrow n \text{ operaciones } O\left( nk \right)$.  

\section{Ejercicio 3. (Alba Bautista)}%
\label{sec:ejercicio_3}
Hemos de calcular $\frac{\sum_{j=1}^{n} c_j}{j}$. Lo hacemos primero para $j = 2^n$. 

Consideramos $C_{in} = \{2^i : 0 \le i \le n\}$ y $\overline{C}_{in} = \left( 1, \ldots, 2^n \right) \ C_n$. 
\[
\sum_{i=1}^{\infty} C_i = \sum_{i \in C_n} C_i + \sum_{i=1}^{n} C_i = \sum_{i=1}^{\infty} 2^i + \sum_{i \in C_n} 1 = 2^{n+1} - 1 + \mid \overline{C}_n \mid \mid \overline{C}_n\mid = \mid 1 \ldots 2^n\mid - \mid C_n \mid = 
\]\[
= 2^n - \left( n + 1 \right) \Rightarrow \sum_{i=1}^{j} C_i = \left( 2^{n + 1} - 1 \right) + \left( 2^n - n - 1 \right) = 3 \cdot 2^n - n - 2 \Rightarrow \frac{\sum_{i=1}^{j} C_i}{2^n} \le 3 \in O\left( 1 \right) 
\]
De forma similar si tomamos $j = 2^n + j': j' \in 1, \ldots, 2^n - 1$.
\[
\Rightarrow \sum_{i=0}^{j} 3\cdot 2^n - n - 2 + j \Rightarrow \frac{\sum_{i=0}^{j} C_i}{j} \le 3 \in O\left( 1 \right)    
\]
Ya que $\begin{cases}
    2^n \le j\\
    j' \le j
\end{cases} $

\section{Ejercicio 4. (Juan Fonseca)}%
\label{sec:ejercicio_4}
Utilizamos el método del potencial para calcular el coste amortizado: 

$\Phi \left( D_i \right)$ elementos en la lista tras operación i-ésima
\begin{itemize}
    \item Añadir un número: 
    \[
    \hat{c}_i = c_1 + \Phi\left( D_i \right) - \Phi\left( D_{i-1} \right) = 1 + \left( k + 1 \right) - k = 2 \in O\left( 1 \right)
    \]
    \item Reducir-lista: 
    \[
    \hat{c}_i = c_1 + \Phi\left( D_i \right) - \Phi\left( D_{i-1} \right) = k + 1 + 1 - k = 2 \in O\left( 1 \right)
    \]
\end{itemize}

\section{Ejercicio 5. (Marta Vicente)}%
\label{sec:ejercicio_5}
\begin{lstlisting}
proc contar(C[0, ..., k - 1] de {0, 1}, E/S posSig: nat)
    j := 0
    mientras j < k AND j < posSig AND C[j] = 1
        C[j] := 0
        j := j + 1
    fmientras

    si j < k AND j < posSig
        C[j] := 1
    fsi

    si j < k AND j = posSig
        C[j] := 1
        posSig := posSig + 1
    fsi
fproc
\end{lstlisting}

\begin{lstlisting}
proc resetear (E/S posSig: nat)
    posSig := 0
fproc
\end{lstlisting}
\[
valor\left( c \right) = \sum_{j=0}^{\text{posSig}} 2^jC\left[ j \right]  
\]

\section{Ejercicio 6. (Juan Diego Barrado)}%
\label{sec:ejercicio_6}
\underline{DISCLAMER}: Este ejercicio no está ni de cerca completo. Recomiendo no utilizar este documento para estudiarlo. 

% Faltan dibujos y la explicación no solo es incompleta sino que también es posiblemente errónea.

\underline{Buscar}:

Para cada $A_i$, hacer búsqueda binaria tiene coste en $O\left( \log 2^i \right) = O\left( i \right)$. 

En el caso peor, el elemento que buscamos está en $A_{k-1}$, luego $\sum_{i=0}^{k-1} i = \frac{\left( k-1 \right) k}{2} \in O\left( k^2 \right) = O\left( \log^2\left( n \right) \right)$.

\underline{Insertar}:
Cuando insertamos $n$ elementos como mucho viajan al nivel $\log\left( n \right)$ vaciamos $A_0$ de $1$ elemento $\frac{n}{2}$ veces, $A_1,\ 2$ elementos $\frac{n}{4}$ veces y $A_j,\ 2^j$ elementos $\frac{n}{2^{j+1}}$ veces $\Rightarrow$
\[
T\left( n \right) = \sum_{j=0}^{\log n} 2^j \cdot \frac{n}{2^{j+1}} = \frac{n}{2} \log n.
\]

\section{Ejercicio 7. (Leonardo Macías)}%
\label{sec:ejercicio_7}
Base: $b\left[ 0, \ldots, n-1 \right] \in \mathbb{Z}^n$ con $b\left[ i \right] \ge 2$.

Contador: $v\left[ 0, \ldots, n-1 \right]$ con $0 \le v\left[ i \right] < b\left[ i \right]$ y $v\left[ 0 \right]$ menos significativo.

\begin{lstlisting}
proc incrementar (b: base, v: contador) 
    i := 0
    mientras i < n AND v[i] = b[i] - 1 hacer
        v[i] := 0
        i := i + 1
    fmientras
    si i < n hacer
        v[i] := v[i] + 1
    fsi
fproc
\end{lstlisting}
$v\left[ 0 \right]$ cambia $n$ veces, $v\left[ 1 \right]$ cambia $\frac{n}{b\left[ 0 \right]}$ veces, \ldots, $v\left[ i \right]$ cambia $\frac{n}{\prod_{j=0}^{i} b\left[ j \right]}$ veces.

\underline{M. Agregación}: 
Nº Cambios: $\sum_{i=0}^{n} \frac{n}{\prod_{j=0}^{i} b\left[ j \right]}$

Como $\forall j \in \{0, \ldots, n-1\}: b\left[ j \right] \ge 2$:
\[
\sum_{i=0}^{n} \frac{1}{\prod_{j=0}^{i} b\left[ j \right]} \le \sum_{i=0}^{n} \frac{1}{2^i} \le \sum_{i=0}^{\infty} \frac{1}{2^i} = 2  
\]

\underline{Coste amortizado}:
\[
\frac{1}{n} \sum_{i=0}^{n} \frac{n}{\prod_{j=0}^{i} b\left[ j \right]} = \sum_{i=0}^{n} \frac{1}{\prod_{j=0}^{i} b\left[ j \right]} \le 2   
\]

\section{Ejercicio 8. (Beñat Pérez)}%
\label{sec:ejercicio_8}
\begin{enumerate}
\item[a)] Utilizamos dos pilas una de los elementos y otra de máximos.

Operaciones:
\begin{lstlisting}
proc apilar(p1, p2: pila; k: ent)
    si !p2.empty AND k <= p2.top =>
        p2.push(k)
    fsi
    p1.push(k)
fproc 
\end{lstlisting}

\begin{lstlisting}
proc maximo (p2: pila; k: ent)
    si !p2.empty => 
        return p2.top
    fsi
fproc
\end{lstlisting}

\begin{lstlisting}
proc desapilar(p1, p2: pila)
    si !p1.empty =>
        si p1.top = p2.top =>
            p2.pop
        fsi
    fsi
    p1.pop
fproc
\end{lstlisting}

\item[b)] Operaciones:

\begin{lstlisting}
proc anyadir(p1: pila; k: ent) 
    si !p1.empty =>
        p1.push(k, max(k, p1.top)
    si no =>
        p1.push(k, max)
    fsi
fproc
\end{lstlisting}

\begin{lstlisting}
func max(p1, p2: pila) 
    return max(p1.top.b, p2.top.b)
fproc
\end{lstlisting}

\begin{lstlisting}
proc desapilar(p1, p2: pilas; k: ent) 
    si !p2.empty =>
        si !p1.empty =>
            p2.push(p1.top.a, p1.top.a)
            p1.pop
            mientras !p1.empty hacer
                p2.push(p1.top.a, max(p1.top.a, p2.top.b)
                p1.pop
            fmientras
            p2.pop
        fsi
    si no =>
        p2.pop
    fsi
fproc
\end{lstlisting}
En el caso peor tendremos $O\left( n \right)$. Sin embargo, amortizado de $n$ operaciones nos saldrá $O\left( 1 \right)$ por operación.

Por el método de contabilidad. Asumamos que apilar tiene coste $3$ (poner el elemento en $p_1$, quitarlo de $p_1$ y ponerlo en $p_2$). Así, solamente nos queda que desapilar tiene coste $1$, ya que habría que hacer una sola operación. 

Además, la función máximo es constante. 

Con todo, nos queda que el coste amortizado es $O\left( 1 \right)$.
\end{enumerate}

\section{Ejercicio 9 (Lucía Alonso)}%
\label{sec:ejercicio_9_lucia_alonso_}
Tenemos una cola $C$, creamos una doble cola llamada $maximos$ para llevar el control del máximo de $C_1$: 
\[
    maximos[i]: 0 \le i < n : \forall j : i < j \le n: maximos[i] \le maximos[j]
\]
\begin{itemize}
    \item En la llamada a $borrar\left( \right)$ elemento de $C_1$ debemos comprobar si es $1^{er}$ máximo.
    \item En la llamada a $insertar\left( \right)$ elemento de $C_1$ debemos borrar todos los elementos de máximo menores que él.
\end{itemize}
\begin{ej}
\begin{align*}
    C = 5, 4, 3, 2, 1;&\quad maximos = 5, 4, 3, 2, 1 \Rightarrow insertar\left( 3 \right)\\
    C = 5, 4, 3, 2, 1, 3;&\quad maximos = 5, 4, 3, 3 \Rightarrow remove\left(\right)\\
    C = 4, 3, 2, 1, 3;&\quad maximos = 4, 3, 3 \Rightarrow insert\left( 6 \right)\\
    C = 4, 3, 2, 1, 3, 6;&\quad maximos = 6
.\end{align*}
\end{ej}
\begin{lstlisting}
proc insertar(cola: C1, doble cola : maximos, elemento: x)
    c.push(x)
    mientras !maximos.empty() AND maximos.back() < x 
        maximo.pop_back()
    fmientras
    maximos.push_back(x)
fproc
\end{lstlisting}

\begin{lstlisting}
proc borrar(cola: C1, doble cola: maximos) 
    si c.front() = maximos.front()
        maximos.pop_front()
    fsi
    c.pop()
fproc
\end{lstlisting}

\begin{lstlisting}
{!maximo.empty()}
func getMax(doble cola: maximo)
    return maximo.front()
ffunc
\end{lstlisting}

\underline{Coste amortizado}. Por el método del potencial.
\begin{itemize}
    \item \underline{Caso peor}: Se añade el máximo de $C$ y $maximos$ está ordenada decrecientemente:
    \[
    c_i = 4 + n \in O\left( n \right) \Rightarrow \text{Lineal.} 
    \]
    En la siguiente llamada, cola vacía: 
    \[
    c_{i+1} = 4 + 1 \in O\left( 1 \right) \Rightarrow \text{cte.} 
    \]
    \item \underline{Método del potencial}: 
    \[
    \Phi\left( D_i \right) = \text{longitud de la doble cola de máximos.} 
    \]
    Al empezar con $C$ vacía se cumple $\forall i: \Phi\left( D_i \right) - \Phi\left( D_{0} \right) > 0$.
    \begin{itemize}
        \item En la llamada a $getMax\left( \right)$: 
        \[
        \hat{c}_i = c_i + \Phi\left( D_i \right) - \Phi\left( D_{i-1} \right) = 1 \in O\left( 1 \right)
        \] Al ser una operación de consulta.
        \item En la llamada $borrar\left(  \right)$: 
        \[
        \hat{c}_i = c_i + \Phi\left( D_i \right) - \Phi\left( D_{i-1} \right) = \begin{cases}
            2 \text{ si no se borra el máximo}\\
            2 + m - 2 -m = 1 \text{ c.c} 
        \end{cases} 
        \]
        \item En la llamada a $insertar\left( \right)$:
        \[
        \hat{c}_i = c_i + \Phi\left( D_i \right) - \Phi\left( D_{i-1} \right) = 4 + b_i + \left( k - b_i \right) - k = 4 \in O\left( 1 \right)
        \]
    \end{itemize}
    Es decir, el coste amortizado es \underline{constante}.
\end{itemize}

\section{Ejercicio 10 (Javier Saras)}%
\label{sec:ejercicio_10}
Por el \underline{método del potencial}. 
\[
\Phi\left( D_i \right) = M + m
\]
Con $\Phi\left( D_0 \right) = 0$ y $\Phi\left( D_i \right) \ge 0$.

\begin{itemize}
    \item $maximo\left( \right)$:
    \[
    \hat{c_i} = 1 + M + m - M - m = 1
    \]
    \item $minimo\left( \right)$:
    \[
    \hat{c_i} = 1 + M + m - M - m = 1
    \]
    \item $eliminar\left( \right)$:
    \begin{itemize}
        \item Máximo: $\hat{c}_i = 2 + M - 1 + m - M - m = 1$.
        \item Mínimo: $\hat{c}_i = 2 + M + m - 1 - M - m = 1$.
        \item Ninguno: $\hat{c}_i = 1 + M + m - M - m = 1$.
    \end{itemize}
    \item $insertar\left( \right)$:
    \begin{itemize}
        \item Máximo: $\hat{c}_i = m + 3 + 1 + m + 1 - \left( M + m \right) = 5$
        \item Mínimo: $\hat{c}_i = m + 3 + M + 1 + 1 - \left( M + m \right) = 5$
        \item Ninguno: $\hat{c}_i = a + b + 3 + M - a + 1 + m - b + 1 - \left( M + m \right) = 5$
    \end{itemize}
\end{itemize}

\section{Ejercicio 11 (Javier Amado)}%
\label{sec:ejercicio_11}
\begin{lstlisting}
class multiconj {
    valores: array <int>
    n: int
    +constructor vacio() {
        valores := new int[max]
        n := 0;
    }

    +insertar(v: int) {
        si (n < max) {
            valores[n] = v
            n := n + 1
        fsi
    }
    +elimMayores {
        nborrar := ceil(n/2)
        m := mediana(valores/n)
        i := 0; elim = 0; j = 0;
        mientras i < n
            si valores[i] > m
                valores [i] := null
                elim := elim + 1
            fsi
            i := i + 1
        fmientras
        mientras j < n AND elim < nborrar
            si valores[j] = m
                valores[j] := null
                elim := elim + 1
            fsi 
            j := j + 1
        fmientras
        i := 0; j := 0
        mientras j < n
            si valores[j] != null
                valores[i] = valores[j]
                i := i + 1
            fsi
        fmientras
        n := n - elim; f elim mayores?
    }
}
\end{lstlisting}
Coste por el método del \underline{potencial} 
\[
\Phi\left( D_i \right) = 2n
\]\[
\hat{c}_{insertar} = c_{insertar} + \left( \Phi\left( D_i \right) - \Phi\left( D_{i-1} \right) \right) = 1 + 2 = 3 \in O\left( 1 \right) 
\]\[
\hat{c}_{elimMayores} = c_{elimMayores} + \left( \Phi\left( D_i \right) - \Phi\left( D_{i-1} \right)\right) = n - 2 \left\lfloor n/2 \right\rfloor = \begin{cases}
    0 \text{ si } n \text{ es par}\\
    1 \text{ si } n \text{ es impar} 
\end{cases} 
\]

\chapter{Hoja 2}%
\label{cha:hoja_2}
\section{Ejercicio 1. (Natalia Rodríguez)}%
\label{sec:ejercicio_1_natalia_rodriguez_}
\begin{lstlisting}
func esEA (t: arbolbinario) dev (h: ent, b: bool)
   caso vacio?(t) entonces (0, true)
       !vacio?(t) entonces sea (l, x, r) = desc(t)
                     (hl, bl) = esEA(l)
                     (hr, br) = esEA(r)
                     en(max(hl, hr) + 1, bl AND br AND |hl - hr| <= 1)
   fcaso
ffunc
\end{lstlisting}

\section{Ejercicio 2. (Dylan Hewitt)}%
\label{sec:ejercicio_2_dylan_hewitt_}
\begin{lstlisting}
func zurdo(T: arbolBinario) dev (esZurdo: bool, nNodos: nat)
  caso vacio?(t) entonces (true, 0)
    !vacio?(t) entonces sea (iz, x, dr) = desc(t) en 
    caso vacio?(iz) AND vacio?(dr) entonces (true, 1)
      !vacio?(dr) OR !vacio?(dr) sean:
              (esZurdoIz, nNodosIz) = zurdo(iz)
              (esZurdoDr, nNodosDr) = zurdo(dr) 
         entonces: 
         (esZurdoDr AND esZurdoIz AND nNodosDr < nNodosIz, 
         nNodosIz + nNodosDr + 1)
    fcaso
  fcaso
ffunc
\end{lstlisting}

\section{Ejercicio 3. (Laura Rodrigo)}%
\label{sec:ejercicio_3_laura_rodrigo}
Dibujo

\section{Ejercicio 4. (Pablo García)}%
\label{sec:ejercicio_4_pablo_garcia_}
(\textit{Dibujo)}

Sea $T_1, T_2$ completos con $h_{T_1} = h_{T_2} + 1$

$T_1, T_2$ son equilibrados por ser completos, $T$ también por serlo $T_1, T_2$ y $\| h_{T_1} - h_{T_2} = \|h_{T_2} + 1 - h_{T_2} \| = 1 \le 1$. Sean $n_{T_i}$ el número de nodos de $T_i$ : 
\begin{align*}
    n_{T_1} &= \sum_{i=0}^{h_{T_1} - 1} 2^{i} = \frac{1 - 2^{h_{T_1}}}{1 - 2} = 2^{h_{T_1}} - 1 = 2^{h_{T_2} + 1} - 1\\ 
    n_{T_2} &= \sum_{i=0}^{h_{T_2} - 1} 2^{i} = \frac{1 - 2^{h_{T_2}}}{1 - 2} = 2^{h_{T_2}} - 1
.\end{align*}
Entonces: 
\begin{align*}
    n_{T_1} - n_{T_2} &= 2^{h_{T_2} + 1} - 1 - 2^{h_{T_2}} + 1\\
                    &= 2^{h_{T_2}} \cdot 2 - 2^{h_{T_2}} \\
                    &= 2^{h_{T_2}} \Rightarrow\\
    \lim_{h_{T_2} \rightarrow \infty} n_{T_1} - n_{T_2} &= \lim_{h_{T_2} \rightarrow \infty} 2^{h_{T_2}} = \infty  
.\end{align*}
Es decir, 
\[
\forall C > 0,\ \exists m \in \mathbb{N} : h_{T_2} \le m \Rightarrow n_{T_1} - n_{T_2} > C.
\]

\section{Ejercicio 5. (Iker Muñoz)}%
\label{sec:ejercicio_5_iker_munoz_}
Supongamos un desequilibrio LL ($h_{iz} = h_{der} + 2$)

(\textit{Dibujo})

Por reducción al absurdo. 
\begin{itemize}
    \item Si $h_{ii} = h_{id} = h \Rightarrow$ No hay desequilibrio.
    \item Si $h_{ii} = h_{id} = h+1 \Rightarrow$ Antes de insertar ya había un desequilibrio, es decir, no se cumplía la precondición.
    \item $h_{id} \neq h_{ii}$ (Análogo RR)
\end{itemize}

Insertamos y hacemos una rotación:

(\textit{Dibujo}) 

La rotación disminuye en $1$ la altura del árbol y tenemos la altura original. 

El elemento ya está insertado y el árbol es AVL, por lo que no es necesario más rotaciones.

\section{Ejercicio 6. (Virginia Chacón)}%
\label{sec:ejercicio_6_virginia_chacón_}
(\textit{Dibujo}) 

\section{Ejercicio 7. (Pablo García)}%
\label{sec:ejercicio_7_pablo_garcia_}
Muy extenso.

\section{Ejercicio 8. (??)}%
\label{sec:ejercicio_8_}
para transformar un AVL en conjunto, lo recorremos en inorden que nos va a dar una lista ordenada. 

\begin{lstlisting}
vector conjunto_arbol(arbol a) 
    vector v;
    inorden(v, a);
    return v;
\end{lstlisting}

\begin{lstlisting}
vector inorden(vector v, arbol a)
    if (a != null) then
        inorden(v, a->izq);
        v.push_back(a->root());
        inorden(v, a->der);
    end if;
\end{lstlisting}

\begin{lstlisting}
vector union_conjunto(const vector& a, const vector& a1)
    int i = 0, j = 0;
    vector union;
    while (i < a
\end{lstlisting}

\section{Ejercicio 9. (Antonio Parrilla)}%
\label{sec:ejercicio_9_antonio_parrilla_}


\section{Ejercicio 22. (Alejandro Ysasi)}%
\label{sec:ejercicio_22_alejandro_ysasi_}
Blanrrojinegro = rojinegro $+$ color azul (solo hijo de rojo).

Ahora tenemos $3$ colores con las siguientes restricciones: 
\begin{itemize}
    \item No puede haber $2$ azules seguidos (nodo azul solo tiene hijos azules).
    \item Un nodo rojo solo puede ser hijo de negro.
    \item No puede haber $2$ rojos seguidos (nodo rojo solo tiene hijo negro o azul).
    \item Azul solo es hijo de rojo.
\end{itemize}

Por tanto, la máxima cantidad de colores en un nodo interno son $1$ negro, $2$ rojos, $4$ azules.

\underline{Jerárquicamente}: negro $>$ rojo $>$ azul.

$2-3-\ldots-8$ árbol permite nodos de hasta $7$ claves y $8$ hijos. 
\begin{itemize}
    \item Si se inserta un $2$-nodo $\Rightarrow 3$ nodo 
    \item $\ldots$
    \item Si se inserta un $8$-nodo $\Rightarrow 2$ nodo $+ 5$ nodo $+ 4$ nodo.
    \item Blanrrojinegro $\Rightarrow$ Ahora se inserta con azul y no con rojo (correspondientes operaciones) 
    \item $2, 3, 4$ nodos igual que rojinegro
\end{itemize}

\chapter{Hoja 3}%
\label{cha:hoja_3}
\section{Ejercicio 5. (María del Mar Ramiro)}%
\label{sec:ejercicio_5_maria_del_mar_}
Montículo de mínimos:
\begin{lstlisting}
fun decrecer(v : monticulo, p : posicion, c : clave)
  si p >= 0 && p < v.size() && c < v[p]
    v[p] := c
    flotar(v, p)
  fsi
ffun
\end{lstlisting}

\begin{lstlisting}
fun aumentar(v : monticulo, p : posicion, c : clave) 
  si p >= 0 && p < v.size() && v[p] < c
    v[p] := c
    hundir(v, p)
  fsi
ffun
\end{lstlisting}

En un montículo de máximos se cambian flotar y hundir.

\section{Ejercicio 7. (Mario Calvarro)}%
\label{sec:ejercicio_7_mario_calvarro_}
En primer lugar, estudiemos la relación entre la representación binaria de la posición de un nodo y el camino de decisiones que nos lleva a dicho nodo (decisión entre tomar el camino izquierdo o el derecho en los nodos que le preceden). Veámoslo con un ejemplo:

\begin{center}
\begin{tikzpicture}[edge from parent/.style={draw,thick},every tree node/.style={draw,thick,circle,align=center, inner sep=0pt, text width=1.5em,text centered,
    font=\sffamily},edge from parent path={(\tikzparentnode) -- (\tikzchildnode)}, sibling distance=.1cm,level distance=10mm]
\Tree [.5
         [.10
           [.14
             [.25 ] 
             [.40 ] ]
           [.21
             [.38 ] 
             \missing ] ]
         [.7 
           [.30 ] 
           [.18 ] ] 
      ]
\end{tikzpicture}
\end{center}

En el árbol presentado tenemos diez nodos diferentes por lo que la representación de sus posiciones en binario irán desde $0001_2$ hasta $1010_2$, es decir, sustituyendo los valores de los nodos por su posición en binario:

\begin{center}
\begin{tikzpicture}[edge from parent/.style={draw,thick},every tree node/.style={draw,thick,circle,align=center, inner sep=0pt, text width=2.2em,text centered,
    font=\sffamily},edge from parent path={(\tikzparentnode) -- (\tikzchildnode)}, sibling distance=.1cm,level distance=10mm]
\Tree [.0001
         [.0010
           [.0100
             [.1000 ] 
             [.1001 ] ]
           [.0101
             [.1010 ] 
             \missing ] ]
         [.0011 
           [.0110 ] 
           [.0111 ] ] 
      ]
\end{tikzpicture}
\end{center}

Lo primero que salta a la vista es que la posición del $1_2$ más significativo representa el ``nivel'' en el que se encuentra el nodo empezando por el nivel 1. Por tanto, si el $1_2$ más significativo se encuentra en el bit menos significativo se tendrá que es el nivel $1$, si está en la siguiente más significativo se tendrá que es el nivel $2$, etc.

Por otro lado, los siguientes bits nos indican las decisiones tomadas en los anteriores niveles, siendo un $0_2$ camino izquierdo y $1_2$ camino derecho. Por ejemplo, si tenemos el nodo $10 = 1010_2$ nos indica que en el camino seguido a la inversa fue izquierda, derecha e izquierda:

\begin{center}
\tikzset{
  verde/.style = {line width=1mm,green}
}
\begin{tikzpicture}[edge from parent/.style={draw,thick},every tree node/.style={draw,thick,circle,align=center, inner sep=0pt, text width=2.2em,text centered,
    font=\sffamily},edge from parent path={(\tikzparentnode) -- (\tikzchildnode)}, sibling distance=.1cm,level distance=10mm]
\Tree [.0001
         \edge[verde];
         [.0010
           [.0100
             [.1000 ] 
             [.1001 ] ]
           \edge[verde];
           [.0101
             \edge[verde];
             [.1010 ] 
             \missing ] ]
         [.0011 
           [.0110 ] 
           [.0111 ] ] 
      ]
\end{tikzpicture}
\end{center}

A su vez, podemos ver que que realizando un desplazamiento hacia la derecha hallamos directamente la posición del padre de un nodo, siempre que los bits que quedan libres a la izquierda los sustituyamos por un $0$. Por ejemplo, si tenemos la posición $10 = 1010_2$ y realizamos un desplazamiento nos quedará $0101_2 = 5$ que es el nodo padre. 

Esta operación es equivalente a la división entera por $2$ que se realiza en un montículo de Williams implementado por vectores.

Con esto podemos demostrar por inducción que la posición $0 < p < m$, con $m = $ bit $1_2$ más significativo, en la cadena de bits nos indica la decisión tomada al bajar del nivel $m - p$:
\begin{itemize}
    \item \underline{Caso base}: Si tenemos un árbol de $3$ elementos la raíz será $01_2$, su hijo izquierdo $10_2$ y el derecho $11_2$. Es decir, el bit menos significativo nos indicará si el camino seguido era el izquierdo ($0$) o derecho ($1$).
    \item \underline{Caso inductivo}: Supongamos ahora que todos los nodos hasta el nivel $p$ cumplen esta propiedad y tomemos un elemento cualquiera del nivel $p + 1$. Si ahora realizamos un desplazamiento a la derecha nos quedará la disposición binaria del nodo del padre por lo que, aplicando la hipótesis de inducción, nos quedará que todos los bits hasta el $p + 1$, que es el $1_2$ que indica el nivel, menos el primero, que es el que se pierde con el desplazamiento, nos indican el camino seguido hasta el padre. Pero
        con el caso base sabemos que el primer bit nos indica si el hijo es el izquierdo o el derecho.
\end{itemize}

Finalmente, para realizar la implementación de la inserción lo que haremos será seguir el camino que nos indica donde se insertará el nuevo elemento manteniendo la estructura de semicompletitud, es decir, el que nos viene dado por $dec2bin(n+1)$\footnote{Suponemos que el tamaño del vector que devuelve será hasta el $1_2$ más significativo.}. 
En el caso de encontrarnos por el camino un nodo con un valor mayor al que se va insertar los cambiaremos y continuaremos con el valor del nodo cambiado. Con esto mantendremos la propiedad de mínimos. Por tanto,

\begin{minipage}{\linewidth}
\begin{lstlisting}[language=c++]
void insertar(const T& x)
{
  //Caso Base
  if (n == 0) 
    t.root() = new Node<T>(x); 
  else 
  {
    vector<bool> camino = dec2bin(n+1); 
    T elemento = x;
    BinTree<T> arbol = t;

    //Como n > 0 => camino.size() >= 2
    for (size_t i = camino.size() - 2; i > 0; --i)
    { 
      if (t.root() > x) 
        swap(arbol.root(), x);

      if (camino[i])
        arbol = arbol->left();
      else 
        arbol = arbol->right();
    }

    if (t.root() > x) 
      swap(arbol.root(), x);

    if (camino[0])
      t.left() = new Node<T>(elem);
    else 
      t.right() = new Node<T>(elem);
  }

  ++n;
}
\end{lstlisting}    
\end{minipage}

El coste del algoritmo vendrá dado por el bucle principal que realiza un total de $\log\left( n \right)$ operaciones (el tamaño del vector de bits será $\log\left( n \right)$) todas ellas constantes. En definitiva, el coste será $O\left( \log\left( n \right) \right)$ en el caso peor.

\section{Ejercicio 8.(Adrián Carlos Skaczylo)}%
\label{sec:ejercicio_8_adrian_carlos_skaczylo_}
Insertar:

\begin{minipage}{\linewidth}
\begin{lstlisting}[language=c++]
flotar(int i) 
  int pos = i;
  Elem p, h, e;
  bool inv = false;
  while (pos != 1 && !inv)
  {
    p = array[pos/2];
    e = array[pos];

    if (pos % 2 = 0 && pos + 1 < array.size()) 
      h = array[pos + 1]
    else 
      h = array[pos - 1]

    if (e.max <= p.min) 
      array[pos] = p
      array[pos/2] = e
    else if (e.min >= p.max) 
      Inv = true
      if (e.max > h.min && e.min < h.max)
        intercambiar(e, h)
    else 
      intercambiar(e, p)
      intercambiar(e, h)

    pos = pos / 2;
  }
\end{lstlisting}
\end{minipage}

Eliminar:

\begin{minipage}{\linewidth}
\begin{lstlisting}[language=c++]
hundir(int i) 
  int pos = i
  Elem e
  int p_HI, p_HD, p_hMP
  bool inv = false;
  p_HI = 2 * pos //puede no existir

  while (!inv && p_HI <= size())
  {
    e = array[pos]
    p_HD = p_HI + 1
    if (p_HD <= size && masPeq(p_HD))
      p_hMP = p_HD
    else 
      p_hMP = p_HI
      p_HD = p_HI

    if (masPeq(e, p_hMP))
      inv = true
    else if(masPeq(p_hMP, e) 
      array[pos] = array[p_hMP]
      array[p_hMP] = e
      intercambiar(p_HI, p_HD)
    else 
      intercambiar(p_hMP, e)
      intercambiar(p_HI, p_HD)

    pos = pos_hMD
    pos_HY = pos * 2
  }
\end{lstlisting}
\end{minipage}

\section{Ejercicio 9. (Pablo Zamora)}%
\label{sec:ejercicio_9_pablo_zamora_}

\section{Ejercicio 11. (Salvador Dzimah)}%
\label{sec:ejercicio_11_salvador_dzimah_}
La secuencia será:
\[
    \begin{matrix} 0 & -1 & 1 & -2 & 2 & -3 & 3 & -4 & 4 \end{matrix} 
\]

\section{Ejercicio 15}%
\label{sec:ejercicio_15}
Árbol semicompleto en el que los niveles de máximos y mínimos se van alternando.

Otra opción es dos árboles de Williams, uno de mínimos y otro de máximos que mantengan una referencia al otro para mantener la consistencia.

Otras opciones:
\begin{itemize}
    \item Deaps.
    \item Máx-mín
    \item Intervalos
\end{itemize}

\section{Ejercicio 23}%
\label{sec:ejercicio_23}

\chapter{Hoja 4}%
\label{cha:hoja_4}
\section{Ejercicio 1}%
\label{sec:ejercicio_1_4}
No tiene sentido porque es no dirigido

\section{Ejercicio 2}%
\label{sec:ejercicio_2_4}
Matriz $V^2$ y lista de adyacencia $V\cdot E$.

\section{Ejercicio 3}%
\label{sec:ejercicio_3_4}
Si es por matriz es el producto normal (en binario). El coste será $V^3$.

Con listas de adyacencia será el coste respecto a $V, E$.

\section{Ejercicio 4}%
\label{sec:ejercicio_4_4}
Pensar con el grado de entrada/salida $0$ o no $0$ y aquellos con ambos $0$.

\section{Ejercicio 5}%
\label{sec:ejercicio_5_4}
Usamos tabla hash de listas. Con la tabla hash solo tenemos la información positiva (actuando como un a matriz). Como los grafos son muy grandes en la práctica, las tablas hash pueden tener muchos conflictos.

\section{Ejercicio 6}%
\label{sec:ejercicio_6_4}
Usamos una matriz de adyacencia. 
Si hay sumidero solo hay uno. Recorremos la matriz buscándolo y, después, lo comprobamos.
Fila todo unos y columna ceros. Al revés sabemos directamente que no es sumidero.

\section{Ejercicio 8}%
\label{sec:ejercicio_8_4}
Dibujos

\section{Ejercicio 9}%
\label{sec:ejercicio_9_4}
Dibujos
Empezamos por el $1$.

\section{Ejercicio 10}%
\label{sec:ejercicio_10_4}
Usamos una pila o una cola dependiendo de si es DFS o BFS.


\section{Ejercicio 11}%
\label{sec:ejercicio_11_4}
Clase.

\section{Ejercicio 12}%
\label{sec:ejercicio_12_4}
Ejercicio 9.9 libro.

\section{Ejercicio 13}%
\label{sec:ejercicio_13_4}
Mismo que antes pero por componentes
Descartamos las aristas que generan un ciclo. 

\section{Ejercicio 14}%
\label{sec:ejercicio_14_4}
9.10 del libro
Grado de salida $> 0$ y el de entrada $= 0$. Solo puede haber uno. 
Primero buscamos los candidatos.

\section{Ejercicio 15}%
\label{sec:ejercicio_15_4}
Cada cambio de adyacencia se cambia el color. Vamos comprobando que se mantienen los colores correctos. Si hay contradicción no es $2$-coloreable.

\section{Ejercicio 16}%
\label{sec:ejercicio_16_4}
Dibujo

\section{Ejercicio 17}%
\label{sec:ejercicio_17_4}
Búsqueda por el camino buscando el mínimo de las aristas. Descartamos aquellos que tienen ``mala'' capacidad.

\section{Ejercicio 18}%
\label{sec:ejercicio_18_4}
Caminos mínimos descartan ciclos. Formamos un subgrafo con las aristas que no forman ciclos. Con esto tenemos un recubrimiento. Con este algoritmo, el camino puede ser mínimo, pero no el árbol.

\section{Ejercicio 19}%
\label{sec:ejercicio_19_4}

\section{Ejercicio 20}%
\label{sec:ejercicio_20_4}
Variación de Dijkstra. No solo calculamos la distancia mínima, también el nº de caminos. Tendremos un vector de distancias y otro de nº de caminos mínimos. Cuando encontramos un camino mejor también tenemos que actualizar el resto de información.

\section{Ejercicio 21}%
\label{sec:ejercicio_21_4}
Otra variación, es caso de igualdad de coste nos quedamos con el camino con menos aristas. Necesitamos llevar información adicional. Debemos mantener correctamente los datos.

\section{Ejercicio 22}%
\label{sec:ejercicio_22_4}
Pensamos en el flujo de los ejercicios anteriores. No interesa la distancia mínima, sino la capacidad mejor.

Nos interesa saber el cuello de botella, si este vale, el resto también.

\section{Ejercicio 23}%
\label{sec:ejercicio_23_4}
Ejercicios 9.11, 9.12 libro.

\section{Ejercicio 24}%
\label{sec:ejercicio_24_4}
Dado árbol de recubrimiento, si empezamos podando las hojas (en ese orden), podemos conseguir lo que se pide.

\section{Ejercicio 25}%
\label{sec:ejercicio_25_4}
Concepto de grafo se extiende con la información adicional de si se puede conectar.

\section{Ejercicio 26}%
\label{sec:ejercicio_26_4}
Usamos como en Kruskal y Prim con colas de prioridad de mínimos. El test que se hace es comprobar cosas (union-find)

\section{Ejercicio 27}%
\label{sec:ejercicio_27_4}
Si es conexo, $\exists$ recubrimiento. Por finitud entre ellos, habrá uno mínimo. Con Kruskal tenemos el bosque de recubrimiento. En el bucle en vez de la condición $V - 1$ vértices debemos cambiarlo en caso de que no tengamos todos los vértices (cola vacía o similar). Similar con Prim. 

\section{Ejercicio 28}%
\label{sec:ejercicio_28_4}
Encontrar un ciclo de longitud  mínima cuyo tamaño sea exactamente $V$.

\section{Ejercicio 31}%
\label{sec:ejercicio_31_4}
El árbol de recubrimiento es único. El orden puede ser distinto. Inducción. Debe cumplir el lema de corte.

\chapter{Hoja 5}%
\label{cha:hoja_5}
Para los ejercicios en general tenemos:
\begin{itemize}
    \item Estrategia
    \item Implementación de la estrategia (normalmente ordenación, en pseudocódigo)
    \item Demostración de la corrección. Solemos usar el método de la reducción de diferencias.
    \item Coste del algoritmo. Suele ser el de la ordenación.
\end{itemize}

\section{Ejercicio 1}%
\label{sec:ejercicio_1_5}
Clases de equivalencia = TAD conjuntos disjuntos. Primero igualdades, después desigualdades. Con igualdades vamos actualizando el Union-Find buscando las clases de equivalencia. Con desigualdades si dos están en la misma clase, son insatisfactibles. Coste lineal.

\section{Ejercicio 2}%
\label{sec:ejercicio_2_5}
Tareas procesador transparencias, generalizado. 12.4 del libro.

\section{Ejercicio 3}%
\label{sec:ejercicio_3_5}
12.5 libro.

\section{Ejercicio 4}%
\label{sec:ejercicio_4_5}
12.10 libro.

\section{Ejercicio 7}%
\label{sec:ejercicio_7_5}
12.8

\section{Ejercicio 8}%
\label{sec:ejercicio_8_5}


\end{document}
